\documentclass{pi1}

\begin{document}

% \maketitle{Nummer}{Abgabedatum}{Tutor-Name}{Gruppennummer}
%           {Teilnehmer 1}{Teilnehmer 2}{Teilnehmer 3}
\maketitle{6}{02.12.2012}{Jan Malburg}{3}
          {Vitalij Kochno}{Yorick Netzer}{Christophe Stilmant}

\section{Der Herr der Tests}
\subsection{Testf\"alle definieren}

\lstinputlisting[firstnumber=1,firstline=1,lastline=142]{../RingBufferTest.java}

\begin{itemize}
	\item Test1: Um zu sehen ob die Array wirklich leer ist.
	\item Test2: Schauen ob die Array gr\"osser wird.
	\item Test3: \"Uberpr\"uft das \"alteste Element.
	\item Test4: F\"ugt eine Liste von 0 bis i dazu, und schaut ob 0 das \"alteste ist.
	\item Test5: Array wird \"uberf\"ullt und die L\"ange sollte aber gleich bleiben
	\item Test6: Hier wollen die Reihefolge \"uberpr\"ufen, ob diese auch eingehalten wird. 
\end{itemize}

\subsection{Implementierung}

\lstinputlisting[firstnumber=16,firstline=16,lastline=200]{../RingBuffer.java}

\section{Spiel's noch einmal, Sam}
Hier definieren wir drei RingBuffer f\"ur x,y und die Rotation.

\lstinputlisting[firstnumber=77,firstline=77,lastline=114]{../Rocket.java}

Anschliessend f\"uhren wir sie aus. Mit ''r'' wird das Replay abgespielt.

\lstinputlisting[firstnumber=155,firstline=155,lastline=166]{../Rocket.java}
	
\end{document}

